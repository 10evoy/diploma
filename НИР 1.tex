\documentclass[12pt,a4paper]{article}

% ---------- ЯЗЫК/КОДИРОВКИ ----------
\usepackage[utf8]{inputenc}
\usepackage[T2A]{fontenc}
\usepackage[russian]{babel}

% ---------- ОФОРМЛЕНИЕ СТРАНИЦЫ ----------
\usepackage{geometry}
\geometry{margin=2.2cm}
\usepackage{setspace}
\onehalfspacing

% ---------- МАТЕМАТИКА/ТАБЛИЦЫ/ССЫЛКИ ----------
\usepackage{amsmath,amssymb}
\usepackage{booktabs}
\usepackage{enumitem}
\usepackage{hyperref}
\hypersetup{colorlinks=true, linkcolor=black, urlcolor=black, citecolor=black}

% ---------- АЛГОРИТМ (МИНИМАЛЬНО) ----------
\usepackage{algorithm}
\usepackage{algpseudocode}

\setlist[itemize]{leftmargin=1.2em}
\setlist[enumerate]{leftmargin=1.2em}

% ---------- ШАПКА ----------
\title{\textbf{Отчёт по НИР}\\
\large Методика построения и реализации индивидуальных траекторий обучения теории вероятностей (7--9 классы) с применением технологий ИИ}
\author{
Вуз: \underline{\hspace{7cm}}\\
Направление подготовки: \underline{\hspace{5.4cm}}\\
Исполнитель: \underline{\hspace{6.4cm}}\\
Научный руководитель: \underline{\hspace{5.6cm}}\\
Город \underline{\hspace{2.8cm}} \quad Год \underline{\hspace{2.0cm}}
}
\date{}

\begin{document}
\maketitle
\thispagestyle{empty}
\newpage

\tableofcontents
\newpage

% ==========================================================
\section*{Аннотация}
В рамках НИР разработана компактная, реализуемая в школьной практике методика построения индивидуальных траекторий обучения теории вероятностей (7--9 классы) с использованием технологий искусственного интеллекта в формате адаптивного назначения заданий. Методика опирается на (1) граф тем (предпосылки $\rightarrow$ следствия), (2) входную диагностику, (3) интерпретируемую по-темную модель уровня освоения ученика, (4) алгоритм выбора следующей темы и сложности, (5) адресную обратную связь по типичным ошибкам, (6) инструменты учителя для мониторинга и минимальный план апробации на двух классах.

\noindent\textbf{Ключевые слова:} индивидуальная траектория, адаптивное обучение, теория вероятностей, knowledge tracing, граф концептов, типичные ошибки, школьная математика.

\newpage

% ==========================================================
\section{Введение}
\subsection{Актуальность}
Раздел «Вероятность и статистика» в 7--9 классах часто сопровождается устойчивыми затруднениями: школьники допускают концептуальные ошибки (ложная равновероятность, путаница независимости и условной вероятности), а класс обычно неоднороден по подготовке. В этих условиях методически оправдана индивидуализация: каждому ученику нужен собственный темп, уровень сложности и набор тренировок на конкретные ошибки.

\subsection{Цель, задачи, гипотеза}
\textbf{Цель НИР:} разработать и описать методику построения и реализации индивидуальных траекторий обучения теории вероятностей (7--9 классы) на основе ИИ-технологий (адаптивное назначение, диагностика ошибок, аналитика прогресса).

\textbf{Задачи:}
\begin{enumerate}
  \item Построить граф тем (concept graph) по вероятности/статистике 7--9 классов.
  \item Разработать схему входной диагностики и инициализации уровня освоения тем.
  \item Определить модель состояния знаний по темам и правила обновления после каждого задания.
  \item Описать алгоритм выбора следующей темы и сложности, систему подсказок и обратной связи.
  \item Подготовить минимальный план апробации методики на двух классах и метрики эффективности.
\end{enumerate}

\textbf{Гипотеза:} если использовать граф тем и по-темную модель освоения с адаптивным подбором заданий и обратной связи по типичным ошибкам, то по итогам апробации повысится результат пост-теста и снизится частота ключевых концептуальных ошибок по сравнению с традиционной практикой.

\subsection{Объект, предмет и методы}
\textbf{Объект:} обучение школьников теории вероятностей (7--9 классы).\\
\textbf{Предмет:} методика проектирования индивидуальных траекторий на основе графа тем и модели по-темного состояния знаний.

\textbf{Методы:} анализ литературы; педагогическое проектирование (граф тем, банк заданий, типология ошибок); моделирование алгоритмов адаптации; апробация (две параллели/два класса); анализ результатов (пред-/пост-тест, сравнение групп).

% ==========================================================
\section{Концептуальная основа методики (кратко)}
\subsection{Knowledge tracing в интерпретируемой постановке}
Для практики важна интерпретируемость: система должна показывать, \emph{какая тема} освоена, а какая нет. Поэтому используем вектор освоения по темам:
\[
M = [M_1,\dots,M_{|V|}], \qquad M_k \in [0,1],
\]
где $M_k$ --- оценка освоения темы $T_k$.

\subsection{Граф тем как «скелет» траектории}
Вводится граф $G=(V,E)$: вершины --- темы, дуги --- зависимости «предпосылка $\rightarrow$ следствие». Граф задаёт допустимый порядок продвижения и позволяет реализовать частичный перенос освоения с базовых тем на связанные.

% ==========================================================
\section{Разработанная методика построения индивидуальной траектории}
\subsection{Этап 1. Граф тем (DAG) по вероятности и статистике 7--9 классов}
\textbf{Принцип построения:} дуга добавляется только при реальной методической необходимости (иначе граф становится «липким», и траектория плохо движется).

\textbf{Рекомендуемый укрупнённый набор тем (пример):}
\begin{itemize}
  \item $T_1$ Базовые понятия: событие, исход, пространство исходов.
  \item $T_2$ Классическая вероятность, равновозможные исходы.
  \item $T_3$ Комбинаторика для подсчёта исходов (правила, перестановки/сочетания по программе).
  \item $T_4$ Диаграммы Эйлера/Венна; операции над событиями.
  \item $T_5$ Правило сложения (в т.ч. пересекающиеся события).
  \item $T_6$ Независимость событий.
  \item $T_7$ Деревья случайных экспериментов.
  \item $T_8$ Условная вероятность (в том числе через дерево/таблицу).
  \item $T_9$ Частота, относительная частота; статистический смысл вероятности.
  \item $T_{10}$ (по необходимости 9 класса) распределение, матожидание/дисперсия на элементарном уровне.
\end{itemize}

\textbf{Минимальная фиксация зависимостей (пример):}
$T_1\rightarrow T_2$, $T_1\rightarrow T_4$, $T_3\rightarrow T_2$, $T_4\rightarrow T_5$, $T_2\rightarrow T_6$, $T_2\rightarrow T_7$, $T_6\rightarrow T_8$, $T_7\rightarrow T_8$, $T_2\rightarrow T_9$.

\subsection{Этап 2. Входная диагностика}
\textbf{Цель:} получить стартовые оценки $M^{(0)}$ по темам и выявить типичные ошибки.

\textbf{Конструкция (минимально):}
\begin{itemize}
  \item 2 задания на тему (одно базовое, одно диагностическое на типичную ошибку);
  \item фиксация: ответ, время, факт запроса подсказки.
\end{itemize}

\textbf{Инициализация:}
\[
M_k^{(0)} = \text{Score}_k = \frac{\#\text{верных по теме }k}{\#\text{заданий по теме }k}.
\]

\subsection{Этап 3. Модель состояния знаний и обновление после каждого задания}
\subsubsection{Обновление по теме задания}
Если ученик решает задание по теме $i$ и получает $r\in\{0,1\}$, то
\[
M_i^{(t+1)} = M_i^{(t)} + \alpha\cdot \delta \cdot (r - M_i^{(t)}),
\quad
M_i^{(t+1)} \leftarrow \min(1,\max(0,M_i^{(t+1)})).
\]
Где $\alpha$ --- скорость обновления (стартово $\alpha=0.2$), $\delta$ --- коэффициент сложности (например $0.8/1.0/1.2/1.4$).

\subsubsection{Перенос по графу предпосылок}
При положительном приросте по теме $i$ он частично переносится на потомков:
\[
M_j^{(t+1)} = M_j^{(t)} + \beta \cdot A_{i,j}\cdot \Delta M_i, \quad (i,j)\in E,
\]
\[
\Delta M_i = \max(0, M_i^{(t+1)}-M_i^{(t)}).
\]
Стартово: $\beta=0.08$, $A_{i,j}=1$ (далее веса можно калибровать).

\subsection{Этап 4. Типичные ошибки как отдельный слой диагностики}
Для теории вероятностей критично диагностировать не только «неверно», но и \emph{почему неверно}. Минимальный перечень ошибок:
\begin{itemize}
  \item \textbf{ложная равновероятность} (equiprobability): «все суммы/исходы равновероятны»;
  \item \textbf{путаница независимости}: вера в влияние предыдущих исходов («ошибка игрока»);
  \item \textbf{путаница формул} объединения/пересечения;
  \item \textbf{путаница $P(A|B)$ и $P(B|A)$}.
\end{itemize}
Практически ошибки удобнее всего распознавать через специально сконструированные distractors (варианты ответа) или по выбранной формуле/шагам решения (если ответы вводятся развернуто).

\subsection{Этап 5. Адаптивный выбор темы и сложности}
\subsubsection{Правило доступности темы}
Тема $k$ доступна, если все её предпосылки выше порога:
\[
\forall p\in Pre(k)\;\; M[p]\ge \theta_{\text{prereq}}, \qquad M[k]<\theta_{\text{mastery}}.
\]
Типично: $\theta_{\text{prereq}}=0.7$, $\theta_{\text{mastery}}=0.8$.

\subsubsection{Алгоритм выбора следующего шага (компактно)}
\begin{algorithm}[H]
\caption{Следующий шаг траектории}
\begin{algpseudocode}[1]
\State $topic \gets$ самая слабая доступная тема ($\min M[k]$ при выполнении предпосылок)
\State $difficulty \gets$ базово по $M[topic]$, затем $\pm 1$ по серии успехов/ошибок
\State выдать задание $(topic,difficulty)$
\State получить ответ $r\in\{0,1\}$; обновить $M$; при необходимости диагностировать тип ошибки
\end{algpseudocode}
\end{algorithm}

\subsection{Этап 6. Подсказки и обратная связь}
\textbf{Подсказки выдаются по запросу} и идут «лесенкой»:
\begin{enumerate}
  \item рефлексивная (наводящий вопрос),
  \item концептуальная (какой принцип применить),
  \item процедурная (план решения),
  \item полное решение (последняя мера).
\end{enumerate}

\textbf{Обратная связь при ошибке} должна ссылаться на типичное заблуждение: например, при ложной равновероятности система кратко показывает число способов получить исход (таблица/перечень), а не просто сообщает правильный ответ.

% ==========================================================
\section{Минимальный план апробации на двух классах}
\subsection{Дизайн апробации}
Планируется апробация на \textbf{двух классах} (пример: 7А и 7Б или 8А и 8Б).
Минимально возможный дизайн:
\begin{itemize}
  \item \textbf{Экспериментальный класс:} занятия + 10--15 минут адаптивной практики 1--2 раза в неделю.
  \item \textbf{Контрольный класс:} занятия по той же теме, но практика традиционная (одинаковый объём).
\end{itemize}
Если администрация/расписание не позволяют «контроль», допустим компромисс: \textbf{до/после} в двух классах с одинаковыми пред- и пост-тестами и сравнением динамики.

\subsection{Материалы апробации (минимум)}
\begin{itemize}
  \item входной пред-тест по темам (20--25 минут);
  \item банк заданий: по 8--12 задач на каждую тему, размеченных по сложности и типичным ошибкам;
  \item пост-тест аналогичной структуры (20--25 минут);
  \item регламент для учителя (как смотреть дашборд и что делать при типовой ошибке).
\end{itemize}

\subsection{Метрики эффективности}
\begin{itemize}
  \item прирост баллов: $\Delta = \text{post} - \text{pre}$ (по сумме и по темам);
  \item доля учеников, достигших $M_k\ge 0.8$ по ключевым темам;
  \item снижение частоты типичных ошибок (по профилю ошибок/по диагностическим заданиям);
  \item процессные метрики: среднее время на задачу, доля запросов подсказок.
\end{itemize}

\subsection{Календарный план (сжатый)}
\begin{center}
\begin{tabular}{@{}p{0.18\linewidth}p{0.76\linewidth}@{}}
\toprule
Срок & Содержание работ \\ \midrule
Нед. 1 & Уточнение тем 7--9 кл.; построение графа; подготовка пред-теста. \\
Нед. 2 & Разметка банка заданий (тема/сложность/ошибка); шаблоны подсказок и обратной связи. \\
Нед. 3--4 & Проведение апробации (2 класса): регулярная практика + сбор данных. \\
Нед. 5 & Пост-тест; выгрузка данных; анализ результатов и корректировка параметров ($\alpha,\beta,\theta$). \\
\bottomrule
\end{tabular}
\end{center}

\subsection{Этика и организация}
\begin{itemize}
  \item Собираются только учебные данные (ответ, время, подсказки), без лишних персональных сведений.
  \item Доступ к аналитике по ролям: ученик видит свой прогресс, учитель --- свой класс.
  \item Результаты апробации используются в обобщённом виде.
\end{itemize}

% ==========================================================
\section{Ожидаемые результаты и выводы}
\subsection{Ожидаемые результаты}
Ожидается, что экспериментальный класс покажет:
\begin{itemize}
  \item больший средний прирост по пост-тесту и по ключевым темам;
  \item уменьшение доли концептуальных ошибок (в первую очередь ложной равновероятности и путаницы формул);
  \item повышение доли учеников, достигших порога освоения по базовым темам.
\end{itemize}

\subsection{Выводы}
Разработанная методика описывает полный, но минимально достаточный цикл индивидуализации: граф тем $\rightarrow$ диагностика $\rightarrow$ по-темная модель освоения $\rightarrow$ адаптивный выбор следующего шага $\rightarrow$ адресная обратная связь по ошибкам $\rightarrow$ учительский мониторинг. Методика ориентирована на внедрение в школе при ограниченных ресурсах и допускает апробацию на двух классах в течение 3--5 недель.

% ==========================================================
\newpage
\begin{thebibliography}{9}

\bibitem{piech2015}
Piech C. et al.
\textit{Deep Knowledge Tracing}.
NeurIPS, 2015.
URL: \url{https://ganguli-gang.stanford.edu/pdf/DeepKnowledgeTracing.pdf}

\bibitem{itswiki}
Intelligent tutoring system (overview).
URL: \url{https://en.wikipedia.org/wiki/Intelligent_tutoring_system}

\bibitem{adaptiveSequencing}
Adaptive curriculum sequencing (overview material).
URL: \url{http://www.italk2learn.eu/wp-content/uploads/2014/04/Adaptive_content_sequencing_18.pdf}

\bibitem{dashboards}
Learning and teaching dashboards for adaptive learning (overview).
URL: \url{https://prometheus-x.org/learning-and-teaching-dashboards-for-adaptive-learning/}

\end{thebibliography}

\end{document}
